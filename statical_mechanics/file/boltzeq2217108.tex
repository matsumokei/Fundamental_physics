%\documentclass[21pt]{jsarticle}\usepackage{ifthen}\newboolean{enlarge}\setboolean{enlarge}{true}
\documentclass[12pt]{jsarticle}\usepackage{ifthen}\newboolean{enlarge}\setboolean{enlarge}{false}
%%%%%%
\setlength{\fullwidth}{16.5truecm}\setlength{\textwidth}{16.5truecm}
\setlength{\oddsidemargin}{0.0truecm}\setlength{\evensidemargin}{0.0truecm}
\setlength{\textheight}{21.5truecm}\setlength{\topmargin}{0.0truecm}
%%%%%%
\usepackage{amsmath,amsfonts,amssymb}
\usepackage[dvipdfmx]{graphicx}
\usepackage[dvipdfmx]{hyperref}\usepackage{pxjahyper}%these two come together
\usepackage{braket}%dirac notation
\usepackage{cases}
\usepackage{here}
\hypersetup{hidelinks}
\interfootnotelinepenalty=10000 % this is to keep a footnote in a single page
\usepackage{bm}%ベクトル記号
\usepackage{ascmac} %囲い
\usepackage{tcolorbox}%tcolorboxの作成
\usepackage{enumerate}
\tcbuselibrary{raster,skins}
\newtheorem{theorem}{定理}
\newtheorem{proof}{証明}
%%%%%%newcomand
\newcommand{\be}{\begin{equation}}
\newcommand{\ee}{\end{equation}}
\newcommand{\bea}{\begin{eqnarray}}
\newcommand{\eea}{\end{eqnarray}}
\newcommand{\nn}{\nonumber \\}

\newcommand{\vnabla}{{\bf \nabla}}
\newcommand{\vsigma}{{\bf \sigma}}
\newcommand{\vr}{{\bm{r}}} %vector r
\newcommand{\vv}{{\bm{v}}} %vector v
\newcommand{\vp}{{\bm{p}}} %vector p
\newcommand{\vphi}{{\varphi(t,\bm{r})}}
\newcommand{\rS}{{\rm{S}}}
\newcommand{\rH}{{\rm{H}}}
\newcommand{\rI}{{\rm{I}}}
%%%%%%
\usepackage{amsmath}
\makeatletter
%%%
%%%  左側、右側に subscript を付記する。
%%%  使い方: \subscripts{左下}{中身}{右下}
%%%
\newcommand{\subscripts}[3]{%
  \@mathmeasure\z@\displaystyle{#2}%
  \global\setbox\@ne\vbox to\ht\z@{}\dp\@ne\dp\z@
  \setbox\tw@\box\@ne
  \@mathmeasure4\displaystyle{\copy\tw@_{#1}}%
  \@mathmeasure6\displaystyle{{#2}_{#3}}%
  \dimen@-\wd6 \advance\dimen@\wd4 \advance\dimen@\wd\z@
  \hbox to\dimen@{}\mathop{\kern-\dimen@\box4\box6}%
}
\makeatother
\makeatletter
% alignL環境 --- 最終行にのみ式番号を振る align 環境
\def\alignL{\csname align*\endcsname}
\def\endalignL{\refstepcounter{equation}\tag{\number\c@equation}%
  \csname endalign*\endcsname}
  % 丸付き文字
\newcommand{\maru}[1]{{\ooalign{\hfil
 \ifnum#1>999 \resizebox{.25\width}{\height}{#1}\else%
 \ifnum#1>99 \resizebox{.33\width}{\height}{#1}\else%
 \ifnum#1>9 \resizebox{.5\width}{\height}{#1}\else #1%
 \fi\fi\fi%
\/\hfil\crcr%
\raise.167ex\hbox{\mathhexbox20D}}}}
%%%%%%
\newcommand{\LS}[2]{\ifthenelse{\boolean{enlarge}}{#1}{#2}}
\newcommand{\LO}[1]{\LS{#1}{\relax}}
\newcommand{\SO}[1]{\LS{\relax}{#1}}
\newcommand{\LNL}{\LO{\notag\\&}}
\renewcommand{\theequation}{%
   \thesection.\arabic{equation}}
  \@addtoreset{equation}{section}
%%%%%%
\begin{document}
\begin{flushright}
\footnotesize
\today
\end{flushright}
\noindent
{\bf
\Large Boltzmann方程式}
\begin{flushright}
理学部第2部 物理学科3年 2217108 松本佳大

\end{flushright}
 位置$\vr$と運動量$\vp$からなる空間を考える.このような空間を位相空間という.この位相空間における分布関数をきめる方程式であるBoltzmann方程式を導出するのが本書の目的である.






%%
\section{偏微分の復習}
二変数関数$f(x,y)$において,$y$を固定し,$x$で微分することを偏微分といい,
\begin{align}
\label{e1}
\frac{\partial f(x,y)}{\partial x}:=\lim_{\Delta x\to 0}\frac{f(x+\Delta x,y)-f(x,y)}{\Delta x}
\end{align}
で定義する.すると,関数$f(x,y)$の微小変化は
\begin{align}
\label{e2}
f(x+\Delta x,y)-f(x,y)=\frac{\partial f(x,y)}{\partial x}\Delta x+O((\Delta x)^2)
\end{align}
とかける.\\
 次に,変数$(x,y)$が$(x+\Delta x,y+\Delta y)$となったときの二変数関数$f(x,y)$の変化を考える.そのために,まず$f(x+\Delta x,y)$を$y$のみの関数とみて,その微小変化を考える.すると,
\begin{align}
f(x+\Delta x,y+\Delta y)-f(x+\Delta x,y)=\frac{\partial f(x+\Delta x,y)}{\partial y}\Delta y+O((\Delta y)^2)\\[5pt]
\label{e3}
\therefore
f(x+\Delta x,y+\Delta y)=f(x+\Delta x,y)+\frac{\partial f(x+\Delta x,y)}{\partial y}\Delta y+O((\Delta y)^2)
\end{align}
となる.次に(\ref{e3})右辺第1項$f(x+\Delta x,y)$を$x$のみの関数とみて,(\ref{e2})を適用すると,
\begin{align}
f(x+\Delta x,y+\Delta y)
=&f(x,y)+\frac{\partial f(x,y)}{\partial x}\Delta x+O((\Delta x)^2)\notag\\[5pt]
&+\frac{\partial f(x+\Delta x,y)}{\partial y}\Delta y+O((\Delta y)^2)
\end{align}
となる.整理すると,
\begin{align}\label{e4}
f(x+\Delta x,y+\Delta y)
=f(x,y)+\frac{\partial f(x,y)}{\partial x}\Delta x&+\frac{\partial f(x+\Delta x,y)}{\partial y}\Delta y\notag\\[5pt]
&+O((\Delta y)^2)+O((\Delta x)^2)
\end{align}
となる.(\ref{e4})右辺の第3項において,$g(x,y):=\dfrac{\partial f(x,y)}{\partial y}$,$g(x+\Delta x,y):=\dfrac{\partial f(x+\Delta x,y)}{\partial y}$とおく.$g(x,y)$を$x$の関数とみて,(\ref{e2})を適用すると,
\begin{align}
\label{e5}
g(x+\Delta x,y)=g(x,y)+\frac{\partial g(x,y)}{\partial x}\Delta x+O((\Delta x)^2)
\end{align}
となる.$g(x,y)$をもとに戻すと,
\begin{align}
\label{e6}
\dfrac{\partial f(x+\Delta x,y)}{\partial y}=\dfrac{\partial f(x,y)}{\partial y}+\frac{\partial }{\partial x}
\left\{
\dfrac{\partial f(x,y)}{\partial y}
\right\}
\Delta x+O((\Delta x)^2)
\end{align}
となる.これを(\ref{e4})右辺の第3項に代入すると,
\begin{align}\label{e7}
f(x+\Delta x,y+\Delta y)
=f(x,y)+\frac{\partial f(x,y)}{\partial x}\Delta x&+
\left(\dfrac{\partial f(x,y)}{\partial y}+\frac{\partial }{\partial x}
\left\{
\dfrac{\partial f(x,y)}{\partial y}
\right\}
\Delta x+O((\Delta x)^2)\right)
\Delta y\notag\\[5pt]
&+O((\Delta y)^2)+O((\Delta x)^2)\notag\\[10pt]
%
=f(x,y)+\frac{\partial f(x,y)}{\partial x}\Delta x&+\dfrac{\partial f(x,y)}{\partial y}\Delta y+\frac{\partial }{\partial x}
\left\{
\dfrac{\partial f(x,y)}{\partial y}
\right\}\Delta x\Delta y\notag\\[5pt]
&
+O((\Delta x)^2))\Delta y+O((\Delta y)^2)+O((\Delta x)^2)\notag\\[5pt]
\end{align}
ここで,2次以上の微小量を無視すれば,
%
\begin{align}\label{e8}
f(x+\Delta x,y+\Delta y)
%
\simeq f(x,y)+\frac{\partial f(x,y)}{\partial x}\Delta x&+\dfrac{\partial f(x,y)}{\partial y}\Delta y
\end{align}
を得る.










%%
\section{Boltzmann方程式}
分布関数$f$を位置$\vr(t)=(r_1(t),r_2(t),r_3(t))$,運動量$\vp(t)=(p_1(t),p_2(t),p_3(t))$,時間$t$の関数$f(\vr(t),\vp(t),t)$の関数とする.時間$\Delta t$経過後の時刻$t+\Delta t$を$t^\prime=t+\Delta t$とおく.そして,時刻$t^\prime$における位置$\vr(t^\prime)$を${\vr}^\prime=\vr(t^\prime)$とおく.また,時刻$t^\prime$における運動量$\vp(t^\prime)$を${\vp}^\prime=\vp(t^\prime)$とおく.ここで,$\vr^\prime,\vp^\prime$がTaylar展開の1次までの項を用いて
\begin{align}\label{b1}
\vr^\prime=\vr(t^\prime)=\vr(t+\Delta t)=\vr(t)+\dfrac{d\vr(t)}{dt}\Delta t=\vr(t)+\Delta \vr(t)\\[10pt]
\label{b2}
\vp^\prime=\vp(t^\prime)=\vp(t+\Delta t)=\vp(t)+\dfrac{d\vp(t)}{dt}\Delta t=\vp(t)+\Delta \vp(t)
\end{align}
と書けることに注意したい.上の2式において,$\Delta \vr(t):=\dfrac{d\vr(t)}{dt}\Delta t$,$\Delta \vp(t):=\dfrac{d\vp(t)}{dt}\Delta t$とおいた.\\
 すると,変数$(\vr,\vp,t)$が$(\vr^\prime,\vp^\prime,t^\prime)$と変化したときの分布関数$f$の変化は
 \begin{align}\label{b3}
 &f(\vr^\prime,\vp^\prime,t^\prime)-f(\vr,\vp,t)=f(\vr+\Delta \vr,\vp+\Delta \vp,t+\Delta t)-f(\vr,\vp,t)\notag
 \intertext{(\ref{e8})より,}\notag\\
 &=\frac{\partial f(\vr,\vp,t)}{\partial r_1}\Delta r_1
+\frac{\partial f(\vr,\vp,t)}{\partial r_2}\Delta r_2
+\frac{\partial f(\vr,\vp,t)}{\partial r_3}\Delta r_3\notag\\[5pt]
&
\ \ \ +\frac{\partial f(\vr,\vp,t)}{\partial p_1}\Delta p_1
+\frac{\partial f(\vr,\vp,t)}{\partial p_2}\Delta p_2
+\frac{\partial f(\vr,\vp,t)}{\partial p_3}\Delta p_3
+\frac{\partial f(\vr,\vp,t)}{\partial t}\Delta t\notag
%
 \intertext{第1,2,3項と第4,5,6項を内積の形に直して}\notag\\
  &=
\left(
\frac{\partial f(\vr,\vp,t)}{\partial r_1}
,\frac{\partial f(\vr,\vp,t)}{\partial r_2},\frac{\partial f(\vr,\vp,t)}{\partial r_3}
\right)\cdot
(\Delta r_1,\Delta r_2,\Delta r_3)\notag\\[5pt]
&
\ \ \ +
\left(
\frac{\partial f(\vr,\vp,t)}{\partial p_1}
,\frac{\partial f(\vr,\vp,t)}{\partial p_2}
,\frac{\partial f(\vr,\vp,t)}{\partial p_3}
\right)
\cdot\left(\Delta p_1,\Delta p_2,\Delta p_3\right)
+\frac{\partial f(\vr,\vp,t)}{\partial t}\Delta t
\end{align}
と書ける.ここで,$\vr$と$\vp$についての微分演算子として,
 \begin{align}\label{nr}
 \nabla_{\vr}&:=
\left(
\frac{\partial }{\partial r_1}
,\frac{\partial }{\partial r_2}
,\frac{\partial }{\partial r_3}
\right)\\[5pt]
\label{np}
 \nabla_{\vp}&:=
\left(
\frac{\partial }{\partial p_1}
,\frac{\partial }{\partial p_2}
,\frac{\partial }{\partial p_3}
\right)
\end{align}
を導入する.また,
  \begin{align}\label{dr}
(\Delta r_1,\Delta r_2,\Delta r_3)
&=\left(
\dfrac{dr_1(t)}{dt}\Delta t,\dfrac{dr_2(t)}{dt}\Delta t,\dfrac{dr_3(t)}{dt}\Delta t
\right)\notag\\[5pt]
&=\frac{d\vr(t)}{dt}\Delta t
\\[10pt]
%
\label{dp}
\left(\Delta p_1,\Delta p_2,\Delta p_3\right)
&=\left(
\dfrac{dp_1(t)}{dt}\Delta t,\dfrac{dp_2(t)}{dt}\Delta t,\dfrac{dp_3(t)}{dt}\Delta t
\right)\notag\\[5pt]
&=\frac{d\vp(t)}{dt}\Delta t
\end{align}
であることを用いると,(\ref{b3})は
  \begin{align}\label{b4}
&f(\vr^\prime,\vp^\prime,t^\prime)-f(\vr,\vp,t)\notag\\[5pt]
&=
\biggl(
\nabla_{\vr}f(\vr,\vp,t)
\biggr)
\cdot
\frac{d\vr(t)}{dt}\Delta t
%
+
\biggl(
\nabla_{\vp}f(\vr,\vp,t)
\biggr)
\cdot
\frac{d\vp(t)}{dt}\Delta t
+
\frac{\partial f(\vr,\vp,t)}{\partial t}\Delta t
\end{align}
(\ref{b4})の両辺を$\Delta t$でわり,整理すれば,
%
\footnotesize
\begin{align}\label{b5}
\frac{\partial f(\vr,\vp,t)}{\partial t}
+
\frac{d\vr(t)}{dt}
\cdot
\biggl(
\nabla_{\vr}f(\vr,\vp,t)
\biggr)
%
+
\frac{d\vp(t)}{dt}
\cdot
\biggl(
\nabla_{\vp}f(\vr,\vp,t)
\biggr)
%
=
\frac
{f(\vr^\prime,\vp^\prime,t^\prime)-f(\vr,\vp,t)}{\Delta t}
\end{align}
\normalsize
さらに,速度$\vv(t)=\dfrac{d\vr(t)}{dt}$とニュートンの運動方程式$\dfrac{d\vp(t)}{dt}=\dfrac{\bm{F}}{m}$を(\ref{b5})へ代入すると
\small
\begin{align}\label{bltzeq}
\frac{\partial f(\vr,\vp,t)}{\partial t}
+
\vv(t)
\cdot
\biggl(
\nabla_{\vr}f(\vr,\vp,t)
\biggr)
%
+
\dfrac{\bm{F}}{m}
\cdot
\biggl(
\nabla_{\vp}f(\vr,\vp,t)
\biggr)
%
=
\frac
{f(\vr^\prime,\vp^\prime,t^\prime)-f(\vr,\vp,t)}{\Delta t}
\end{align}
\normalsize
の式を得る.この式は分布関数$f(\vr,\vp,t)$をきめるべき基本的な方程式であり,Boltzmann方程式という.(\ref{bltzeq})の右辺は分布関数の変化が生じる原因を示す項である.
















%%%%%%%%%%%%%%%%%%%%%%
 \begin{thebibliography}{99}

\item
{阿部 龍蔵(1969)
『電気伝導(新物理学シリーズ 8)』(培風館)}
\item
{鈴木 彰・藤田 重次(2008) 『統計熱力学の基礎』(共立出版)}
\item
{笠原 晧司(2006)
『対話・微分積分学―数学解析へのいざない』(現代数学社)}
\end{thebibliography}
\end{document}


 