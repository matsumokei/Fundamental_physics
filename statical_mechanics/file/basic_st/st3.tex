\part{}
\section{平衡状態の本質}
\subsection{等重率の原理}
平衡状態のマクロな性質とは「典型的な状態」が共通にもっている性質のことである.我々が共通の性質を知りたいのならば,方針として,「例外的状態を含めて,取りうる状態すべてについて平均をとる」ことである.この方針は例外的な状態を含めても,圧倒的大多数が典型的な状態であるから,平均的な性質に与える影響は無視できるはずである.つまり,等確率になるモデルを考えれば,平均を考えることができる.そこで,次に示す等重率の原理\footnote{%
原理と名前についているが,本当に成立しているかどうかは証明できない.
}
あるいは等確率の原理と呼ばれる確率モデルを考える.
 \begin{itembox}[l]{等確率の原理}
「エネルギーが$E$となる各力学的な状態は,等確率で出現する.」
\end{itembox}
これは,「共通の性質」を抽出するための理論的方策である.
\section{小正準集合の方法}
\subsection{ミクロカノニカルアンサンブル}
 \begin{itembox}[l]{小正準集合(ミクロカノニカルアンサンブル)}
$E$に$[E,E+\delta E]$と幅を持たせる.小正準集合とは「$N$,$V$,$E\sim E+\delta E$を指定したときに,各々の取りうる力学的状態が等確率(定数)で出現する確率モデル」である.
\end{itembox}
上の文章をそのまま式で表そう.状態$(N,V,[E+\delta E])$を指定したときにある力学的な状態「$r$」が出現する確率を$p_{N,V,E}(r)$とすると
\begin{equation}
p_{(N,V,E)}(r)=A\mbox{(定数)}
\end{equation}
とかける.すべての力学状態に対する足し上げを行うと
  \begin{eqnarray*}
\begin{split}
\displaystyle\sum_{r}p_{(N,V,E)}(r)=\displaystyle\sum_{r}A=A\displaystyle\sum_{r}1
  \end{split}
\end{eqnarray*}
$\displaystyle\sum_{r}1$は全状態数になっているから,$(N,V,[E+\delta E])$をとる状態数を$W(N,V,E)$とおくと,
  \begin{eqnarray*}
\begin{split}
\displaystyle\sum_{r}p_{(N,V,E)}(r)=A\displaystyle\sum_{r}1=A\ W(N,V,E)
  \end{split}
\end{eqnarray*}
となる.これが1と等しいから,
  \begin{eqnarray*}
\begin{split}
A=\frac{1}{W(N,V,E)}
  \end{split}
\end{eqnarray*}
となり,
\begin{equation}
\label{p1}
p_{(N,V,E)}(r)=
{}
\begin{cases}
\dfrac{1}{W(N,V,E)}\\[20pt]
0
    \end{cases}
\end{equation}
を得る.
\subsection{状態数$W(N,V,E)$とエントロピー}
粒子数$N$,体積$V$,エネルギー$E$の孤立系の取りうる全状態数を$W(N,V,E)$とする.このとき,平衡状態を与える力学的状態の状態数$W_{\rm{eq}}(N,V,E)$とそれ以外の状態を与える力学的状態の状態数$W_{\rm{neq}}(N,V,E)$および,全状態数$W(N,V,E)$との間には
\begin{align}
\label{w}
W(N,V,E)=W_{\rm{eq}}(N,V,E)+W_{\rm{neq}}(N,V,E)
\end{align}
が成立する.
取りうる力学的状態の圧倒的大多数が,マクロな物理量から見て,平衡状態が区別がつけない状態(典型的状態)で占められる.すなわち,
\begin{align}
W_{\rm{eq}}(N,V,E)\gg W_{\rm{neq}}(N,V,E).
  \end{align}
%%%%%%%%%
これにより,(\ref{w})は
\begin{align}
\label{w1}
W(N,V,E)\simeq W_{\rm{eq}}(N,V,E)
  \end{align}
  となる.
  \paragraph{孤立系のエントロピー}
  孤立系で変化が起きるとエントロピー$S$は増大,または一定となる.エントロピー	$S$は非現象的な量であり,エントロピー$S$が最大のとき,平衡状態となる.よって,平衡状態の$S$と$W_{\rm{eq}}(N,V,E)$は次のように関係づけれるだろう.
\begin{align}
\label{S1}
S=f\biggl(W_{\rm{eq}}(N,V,E)\biggr)\simeq f\biggl(W(N,V,E)\biggr)
  \end{align}
  エントロピーの相加性の性質より
  \begin{align}
\label{S2}
S_{\rm{A+B}}=S_{\rm A}+S_{\rm B}.
  \end{align}
  ここで,$S_{\rm{A}}$は系Aのエントロピー,$S_{\rm{B}}$は系Bのエントロピーであり,系Aと系Bを足した系A$+$Bのエントロピーを$S_{\rm{A+B}}$で表した.\\
 状態数$W(N,V,E)$の場合は系Aの状態数を$W_{\rm{A}}$,系Bの状態数を$W_{\rm{B}}$とすると,系Aと系Bを足した系A$+$Bの状態数$W_{\rm{A+B}}$は
   \begin{align}
\label{w2}
W_{\rm{A+B}}=W_{\rm A}\times W_{\rm B}.
  \end{align}
  となる.(\ref{S1})より,
  \begin{eqnarray*}
\begin{split}
S_{\rm{A+B}}=f\biggl(W_{\rm{A+B}}(N,V,E)\biggr)
  \end{split}
\end{eqnarray*}
上式の左辺は(\ref{S2})より,
\begin{eqnarray*}
\begin{split}
S_{\rm{A+B}}=S_{\rm A}+S_{\rm B}=f\biggl(W_{\rm{A}}(N,V,E)\biggr)+f\biggl(W_{\rm{B}}(N,V,E)\bigg)
  \end{split}
\end{eqnarray*}
となり,上式の右辺は
 \begin{eqnarray*}
\begin{split}
f\biggl(W_{\rm{A+B}}(N,V,E)\biggr)=f\biggl(W_{\rm{A}}(N,V,E)\times W_{\rm{B}}(N,V,E)\biggr)
  \end{split}
\end{eqnarray*}
となるから,
\begin{align}
f\biggl(W_{\rm{A}}(N,V,E)\biggr)+f\biggl(W_{\rm{B}}(N,V,E)\bigg)=f\biggl(W_{\rm{A}}(N,V,E)\times W_{\rm{B}}(N,V,E)\biggr)
\end{align}
が得られる.
となるが,(\ref{e5})と比較すると
\begin{equation}
\label{eA}
\displaystyle\sum_{i=1}^3a_{ij}a_{ik}=\delta_{jk} .\ \ \ \ \ \ j,k=1,2,3\tag{A}
\end{equation}
が要求される.\\
 次に,$S^\prime$系から$S$系への変換,すなわち逆変換を考えよう.逆変換が次のようにかけるとする.
  \begin{eqnarray*}
\begin{split}
\displaystyle\sum_{j=1}^3a_{jk}x^\prime_j
=\displaystyle\sum_{i=1}^3\displaystyle\sum_{j=1}^3a_{jk}a_{ji}x_i
=\displaystyle\sum_{i=1}^3\delta_{ki}x_i=x_k
  \end{split}
\end{eqnarray*}
すなわち,
\begin{align}
x_k=\displaystyle\sum_{j=1}^3a_{jk}x_j^\prime
  \end{align}
により与えられる.そこで,
  \begin{eqnarray*}
\begin{split}\displaystyle\sum_{k=1}^3x_k x_k
&=
\left(
\renewcommand{\arraystretch}{1.5}
    \begin{array}{c}
  x_1\\
  x_2\\
    x_3\\
    \end{array}
  \right)\cdot
  \left(
\renewcommand{\arraystretch}{1.5}
   \begin{array}{c}
  x_1\\
  x_2\\
    x_3\\
    \end{array}
  \right)\\[20pt]
  &=%
  \left(
\renewcommand{\arraystretch}{1.5}
    \begin{array}{c}
     \displaystyle\sum_{j=1}^3a_{j1}x_j^\prime\\
  \displaystyle\sum_{j=1}^3a_{j2}x_j^\prime\\
    \displaystyle\sum_{j=1}^3a_{j3}x_j^\prime\\
    \end{array}
  \right)\cdot
  \left(
\renewcommand{\arraystretch}{1.5}
    \begin{array}{c}
  \displaystyle\sum_{i=1}^3a_{i1}x_i^\prime\\
  \displaystyle\sum_{i=1}^3a_{i2}x_i^\prime\\
    \displaystyle\sum_{i=1}^3a_{i3}x_i^\prime\\
    \end{array}
  \right)\\[20pt]
  &=
\Biggl(\displaystyle\sum_{j=1}^3a_{j1}x_j\Biggr)\Biggl(\displaystyle\sum_{i=1}^3a_{i1}x_i\Biggr)
+\Biggl(\displaystyle\sum_{j=1}^3a_{j2}x_j\Biggr)\Biggl(\displaystyle\sum_{i=1}^3a_{i2}x_i\Biggr)
+\Biggl(\displaystyle\sum_{j=1}^3a_{j3}x_j\Biggr)\Biggl(\displaystyle\sum_{i=1}^3a_{i3}x_i\Biggr)\\[10pt]
&=\displaystyle\sum_{j=1}^3\displaystyle\sum_{k=1}^3\Biggl(\displaystyle\sum_{i=1}^3a_{kj}a_{ki}\Biggr)x_jx_i
  \end{split}
\end{eqnarray*}
となり,(\ref{e5})より,
\begin{equation}
\label{eB}
\displaystyle\sum_{i=1}^3a_{jk}a_{ki}=\delta_{ij} .\ \ \ \ \ \ i,j=1,2,3\tag{B}
\end{equation}
が要求される.\\
%%%%%%%%%%%%%%%%%%%%%
%%%%%%%%%%%%%%%%%%%%%
このことから,旧座標と回転した座標が直交座標系のとき,
 \begin{numcases}
{}
\label{A}
\displaystyle\sum_{i=1}^3a_{ij}a_{ik}=\delta_{jk} .\ \ \ \ \ \ j,k=1,2,3&\\[10pt]
\label{B}
\displaystyle\sum_{i=1}^3a_{jk}a_{ki}=\delta_{ij} .\ \ \ \ \ \ i,j=1,2,3&
\end{numcases}
であることが要請されることがわかる.\\
 (\ref{A})は,行列Aの列の成分が互いに直交し,自分自身との内積が1になることを表している.また,(\ref{B})は行列Aの転地行列${}^tA$の列の成分が互いに直交し,自分自身との内積が1になることを表している.\\
 (\ref{A}),(\ref{B})の性質から,
 \begin{align}
 \label{E}
 {}^tAA=E
 \end{align}
 であることがわかる.(\ref{E})の性質を満たす,正方行列$A$を直交行列という.
 
