\part{grand canonical distribution}
\section{大正準集合(グランドカノニカルアンサンブル)の方法}
\subsection{}
全系の取りうる状態数を$W(N_{\rm T},E_{\rm T})$とおく.\footnote{%
TはTotalの意味である
}
いま,注目系が粒子数$N$をとり,ある一つの状態$i$を取っているとする.そのときのエネルギーを$E_{N,i}$とする.一つの状態$i$を指定したときの「熱浴」の取りうる状態を
 \begin{eqnarray*}
  \begin{split}
W_{\rm B}(N_{\rm B},E_{\rm B})=W_{\rm B}(N_{\rm T}-N,E_{\rm T}-E_{N,i})
  \end{split}
\end{eqnarray*}
とする.ここで,$N_{\rm B}=N_{\rm T}-N$,$E_{\rm B}=E_{\rm T}-E_{Ni}$である.\\
 すると,全系の状態数は$W_{\rm B}$を全粒子数かつ全状態で和を取ったものであり,
 \begin{align}
W(N_{\rm T},E_{\rm T})=\displaystyle\sum_{N}\displaystyle\sum_{i}W_{\rm B}(N_{\rm T}-N,E_{\rm T}-E_{N,i})
\end{align}
となる.\\
 注目系が,粒子数$N$,ある一つの状態$i$をとる確率$P_{N,i}$は
 \begin{align}
P_{N,i}=
\frac{W_{\rm B}(N_{\rm T}-N,E_{\rm T}-E_{Ni})}{\displaystyle\sum_{N}\displaystyle\sum_{i}W_{\rm B}(N_{\rm T}-N,E_{\rm T}-E_{N,i})}
\end{align}
という関係式において,電場$\bm E_0$は$z$方向のみに依存するから,
\begin{eqnarray*}
\begin{split}
\frac{d}{dz}\phi_0=-E_0
  \end{split}
\end{eqnarray*}
となり,これを解くと,
\begin{eqnarray*}
\begin{split}
\phi_0=-E_0z+C
  \end{split}
\end{eqnarray*}
となる.そして,$z=r\cos\theta$であるから,静電ポテンシャル$\phi_0$は
 \begin{align}
\phi_0=-E_0r\cos\theta+C
\end{align}
で与えられる.\\
 次に導体の外部での静電場$\bm E_1$による,静電ポテンシャル$\phi_1(x,y,z)$を求めよう.これを求めるには次のLaplaceの方程式を解けばいい.
 \begin{align}
\Delta\phi_1(x,y,z)=0
\end{align}
