\documentclass[12pt]{jsarticle}\usepackage{ifthen}\newboolean{enlarge}\setboolean{enlarge}{false}
%%%%%%
\setlength{\fullwidth}{16.5truecm}\setlength{\textwidth}{16.5truecm}
\setlength{\oddsidemargin}{0.0truecm}\setlength{\evensidemargin}{0.0truecm}
\setlength{\textheight}{21.5truecm}\setlength{\topmargin}{0.0truecm}
%%%%%%
\usepackage{amsmath,amsfonts,amssymb}
\usepackage{graphicx}
\usepackage[dvipdfmx]{hyperref}
\usepackage{pxjahyper}%these two come together
\usepackage[dvipdfmx]{color}
\usepackage{braket}%dirac notation
\usepackage{wrapfig}
\usepackage{here}
\usepackage{tabularx, dcolumn}
\usepackage{subfigure}
\usepackage{cases}
\usepackage{bigints}%インテグラルで大きくする
\usepackage{mathtools} 
\hypersetup{hidelinks}
\interfootnotelinepenalty=10000 % this is to keep a footnote in a single page
\usepackage{bm}%ベクトル記号
\usepackage{ascmac} %囲い
%%%%%%%%%%%%%%%%%%%%%%%%%%%%%%%%%%%%%%
%\usepackage{txfonts} %設定部分
\usepackage{listings, jlisting}
\renewcommand{\lstlistingname}{リスト}
\lstset{language=c,
basicstyle=\ttfamily\scriptsize,
commentstyle=\textit,
classoffset=1,
keywordstyle=\bfseries,
frame=tRBl,
framesep=5pt,
showstringspaces=false,
numbers=left,
stepnumber=1,
numberstyle=\scriptsize,
tabsize=2,
breaklines = true,
}

\lstnewenvironment{mylisting}[1][]
    {\lstset{
        frame=single,
        basicstyle=\ttfamily,
        numbers=left,
        numbersep=10pt,
        tabsize=2,
        extendedchars=true,
        xleftmargin=17pt,
        framexleftmargin=17pt,
        #1
    }
}{}
\usepackage{listings, jlisting, color}
\definecolor{OliveGreen}{rgb}{0.0,0.6,0.0}
\definecolor{Orenge}{rgb}{0.89,0.55,0}
\definecolor{SkyBlue}{rgb}{0.28, 0.28, 0.95}
\lstset{
  language={python}, % 言語の指定
  basicstyle={\ttfamily},
  identifierstyle={\small},
  commentstyle={\smallitshape},
  keywordstyle={\small\bfseries},
  ndkeywordstyle={\small},
  stringstyle={\small\ttfamily},
  frame={tb},
  breaklines=true,
  columns=[l]{fullflexible},
  numbers=left,
  xrightmargin=0zw,
  xleftmargin=3zw,
  numberstyle={\scriptsize},
  stepnumber=1,
  numbersep=1zw,
  lineskip=-0.5ex,
  keywordstyle={\color{SkyBlue}},     %キーワード(int, ifなど)の書体指定
  commentstyle={\color{OliveGreen}},  %注釈の書体
  stringstyle=\color{Orenge}          %文字列
}
%%%%%%%%%%%%%%%%%%%%%%%%%%%%%%%%%%%%%
\usepackage{tikz}
\usepackage{amsmath}
\usepackage{cases}%連立方程式


\usepackage{mdframed}%ページをまたぐ    
\newmdenv[skipabove=6mm, skipbelow=4mm]{kotak}



\newtheorem{definition}{定義}[section]
\newtheorem{theorem}[definition]{定理}
\newtheorem{proof}{証明}
\newtheorem{request}[definition]{要請}
\newtheorem{these}[definition]{仮定}
\newtheorem{lemma}[definition]{補題}
\newtheorem{postlate}[definition]{公理}



%%%%%%newcomand
\newcommand{\be}{\begin{equation}}
\newcommand{\ee}{\end{equation}}
\newcommand{\nn}{\notag \\}


\newcommand{\ha}{\hat{a}}
\newcommand{\hO}{\hat{\Omega}}
\newcommand{\hpsi}{\hat{\psi}}
\newcommand{\hV}{\hat{V}}
\newcommand{\vac}{\ket{{\rm{vac}}}}
\newcommand{\hU}{\hat{U}}
\newcommand{\hK}{\hat{K}}
\newcommand{\hH}{\hat{H}}
\newcommand{\hB}{\hat{B}}
\newcommand{\hA}{\hat{A}}
\newcommand{\hp}{\hat{p}}
\newcommand{\hdA}{\Delta\hat{A}}
\newcommand{\hrho}{\hat{\rho}}

\newcommand{\Tr}{{\rm{Tr}}}

\newcommand{\kb}{k_{\rm{B}}}


%ボソン
\newcommand{\sk}{{\ket{\alpha_{a}\alpha_{b}\cdots\alpha_{g}}_{\rm S}}}
\newcommand{\hS}{\hat{S}}
\newcommand{\hP}{\hat{P}}
\newcommand{\hn}{\hat{n}}

%
\newcommand{\ep}{\epsilon} 
\newcommand{\vr}{{\bm{r}}} %vector r
\newcommand{\vq}{{\bm{q}}} %vector q
\newcommand{\vp}{{\bm{p}}} %vector p
\newcommand{\vk}{{\bm{k}}} %vector k
\makeatletter
% alignL環境 --- 最終行にのみ式番号を振る align 環境
\def\alignL{\csname align*\endcsname}
\def\endalignL{\refstepcounter{equation}\tag{\number\c@equation}%
  \csname endalign*\endcsname}
%%%%%%
\def\trmatrix#1{%
  \sideset{^t\!\!}{}{\mathop{\begin{pmatrix}#1\end{pmatrix}}}
}
%%%%%%
\renewcommand{\theequation}{%
   \thesection.\arabic{equation}}
  \@addtoreset{equation}{section}
\begin{document}
\begin{flushright}
\footnotesize
\today
\end{flushright}
\noindent
{\bf
\Large 常微分方程式の数値計算手法}
\begin{flushright}
東京理科大学大学院理学研究科物理学専攻修士二年
1221544 松本佳大
\end{flushright}


\section{2種類の誤差}
まず,数値計算を行う際に生ずる誤差について説明する.主に以下の2種類の誤差がある.
\begin{kotak}
	\begin{description}
   \item[丸め誤差:]コンピュータが実数を有限の小数に変換する際に生じる誤差.
   \item[打ち切り誤差:]級数など無限に続く式や極限をとる操作を,計算のために途中で打ち切ったときに生ずる誤差.
\end{description}
\end{kotak}
本書では,この打ち切り誤差について考えることにする.

\section{オーダー記法について}
$\mathcal{O}$記法は,$n$の値を増やしたときに,関数$f(n)$の値がどの程度の勢いで増えていくかを表現するものである.増加率の高い項のみを残し,

\section{常微分方程式の初期値問題と解の離散近似}
$x(t)$を時間$t$に関する変数とし,その時間変化が$f(x,t)$で与えられるとする.
\begin{equation}\label{ODE}
	\left\{
\begin{array}{l}
\dfrac{d}{dt}x(t)=f(x(t),t) \\[10pt]
x(t_0)=x_0
\end{array}
\right.
\end{equation}
ここで,$x(t_0)=x_0$は$x$の時刻$t_0$での初期値であり,初期条件とよばれる.このように与えられた初期条件のもとで状微分方程式を解く問題を初期値問題とよぶ.
まず,変数$t$の範囲を有界区間$t_0\leq t\leq T$に限定し(ただしどこまで計算するのか,すなわち$T>t_0$は事前に決める),初期値問題\eqref{ODE}について考える.ここで,関数$f(x)$および初期値$t_0,x_0$は与えられているものとする.これは初期値問題の解を時刻$T$まで求めよという問題である.そこで,$N$を正の整数として
\begin{equation}
	t_k=t_0 + k \Delta t\ \ \ \ (k=0,1,\ldots,N),\ \ \ \ \ \Delta t=\frac{T-t_0}{N}
\end{equation}
とおき,時刻$t_0,t_1,\ldots,t_N=T$のみで解の値$x(t_i)$を考えることにする.これを解の離散近似という.$\Delta t$を刻み幅,ステップ数と呼ぶ.一般には$N$の値を十分大きくとり,解$x(t)$を精度よく近似できるようにする.ただし,$N$の値が大きすぎるとステップが増大し,計算時間の増加につながる.また,丸め誤差の影響で逆に精度が悪くなる.
%
\section{Euler Method}
\eqref{ODE}を数値的に解く最も簡単な方法が(陽的)Euler方である.関数$x(t)$において微分を次のように定義する:
\begin{equation}
	\frac{dx(t)}{dt}=\lim_{\Delta t\to0}\frac{x(t+\Delta t)-x(t)}{\Delta t}
\end{equation}
すると,関数$x(t)$の微小変化は
\begin{equation}
	x(t+\Delta t)-x(t)=\frac{dx(t)}{dt}\Delta t+\mathcal{O}((\Delta t)^2)
\end{equation}
と書けるので,$x(t)$の微分は次のように近似できる:
\begin{equation}
	\frac{dx(t)}{dt}\simeq\frac{x(t+\Delta t)-x(t)}{\Delta t}
\end{equation}
したがって,微分方程式を
\begin{equation}
	\frac{x(t+\Delta t)-x(t)}{\Delta t}
	\simeq f(x(t),t)
\end{equation}
と近似できる.そこで,時刻$t_n=n\Delta t$での解$x(t_n)$の近似値を$x_n$とおく.近似値$x_n$を等式
\begin{equation}
	\frac{x_{n+1}-x_n}{\Delta t} = f(x_n,t_n)
\end{equation}
を満たすようにとる.これを変形して,次のように書く:
\begin{align}\label{Euler method}
	x_{n+1}&=x_n+k_1\\[10pt]
	k_1&=\Delta t f(x_n, t_n)
\end{align}

この式に従い,逐次的に$x_k$,$k = 1,2,\ldots,N$を求める方法をEuler法と呼ぶ.ここで,初期値$x(0)=x_0$は与えられているので,式\eqref{Euler method}を用いることで,$x(t)$の値を$x(0)$,
\begin{align}
	x(\Delta t)\simeq x_1&=x(t=0) +\Delta t\ f(x_0,t=0)\\[10pt]
	x(2\Delta t)\simeq x_2&=x(\Delta t) +\Delta t\ f(x_1,\Delta t)\\[10pt]
	&\hspace{30pt}\vdots\nn
	&\hspace{30pt}\vdots\notag
\end{align}
と$\Delta t$の刻み幅で逐次的に計算することが可能である.


\subsection{Euler法の誤差解析}
$x(t+\Delta t)$の$t$の周りでのテイラー展開を考えると,
\begin{align}
	x(t+\Delta t)&=x(t) +\frac{1}{1!}x^{\prime}\Delta t +\frac{1}{2!}x^{\prime\prime}\Delta t^2+\cdots\nn[10pt]
	&=x(t) +\frac{1}{1!}f(x,t)\Delta t +\frac{1}{2!}f^{\prime}(x,t)\Delta t^2+\cdots\nn[10pt]
	&=x(t) +\frac{1}{1!}f(x,t)\Delta t +\mathcal{O}((\Delta t)^2)
\end{align}
である.Euler法の局所誤差を評価する.局所誤差$\epsilon_k$,$k=1,2,\ldots,N$を時刻$t_{k}$における$x(t_k)$とその近似解$x_k$と差で定義する:
\begin{equation}
	\epsilon_{k}\equiv x(t_{k})-x_k
\end{equation}
Eulerの差分公式\eqref{Euler method}
\begin{align}
	x_{n+1}&=x_n+\Delta t f(x_n, t_n)
\end{align}
において,近似値$x_n$に真の解$x(t_n)$を代入すると
\begin{align}
	x_{n+1}&=x(t_n)+\Delta t f(x(t_n), t_n)
\end{align}
を得る.これは$x_{n+1}$の一つ前の解に真の解を用いていることに対応する.したがって局所誤差$\epsilon_{n+1}$は
\begin{align}
	\epsilon_{n+1}&=x(t_{n+1})-x_{n+1}\nn[10pt]
	&=x(t_n+\Delta t)-x(t)-f(x(t_n), t_n)\nn[10pt]
	&=
	\frac{1}{1!}x^\prime(t_n)\Delta t +\frac{1}{2!}x^{\prime\prime}(t_n)\Delta t^2+\mathcal{O}(\Delta t^3)
	-f(x(t_n),t_n)\nn[10pt]
	&=f(x(t_n),t_n)+\frac{1}{2!}x^{\prime\prime}(t)\Delta t^2 -f(x(t_n),t_n)+\mathcal{O}(\Delta t^3)\\[10pt]
	&=\frac{1}{2!}x^{\prime\prime}(t)\Delta t^2+\mathcal{O}(\Delta t^3)
	=\mathcal{O}(\Delta t^2)
\end{align}
となり,$\Delta t^2$に比例する.することがわかる.次に,$n$ステップ後に蓄積した誤差$\epsilon^{\rm{gb}}$を計算する.$N=n\Delta t$より,$n \propto \Delta t^{-1}$,$n=\mathcal{O}(\Delta t)$であるから,$\epsilon^{\rm{gb}}$は
\begin{equation}
	\epsilon^{\rm{gb}}=n\epsilon
	=\mathcal{O}(\Delta t^{-1})\mathcal{O}(\Delta t^2)
	=\mathcal{O}(\Delta t)
\end{equation}
と求まる.



\section{ホイン法}
\begin{equation}
	x_{n+1}=x_n+\frac{\Delta t}{2}\left(
	f(x_n,t_n)
	+f(\bar{x}_{n+1},t_{n}+\Delta t)
	\right)
\end{equation}
である.ここで,近似値$\bar{x}_{n+1}$をEuler法を用いて計算した($\bar{x}_{n+1}=x_n+\Delta t f(x_n,t_n)$)とすると,次の公式を得る:
\begin{align}\label{Heun}
	x_{n+1}&=x_n+\frac{\Delta t}{2}(k_1+k_2)\\[10pt]
	k_1&=f(x_n,t_n)\\[10pt]
	k_2&=f(x_n+\Delta t k_1, t_n + \Delta t)
\end{align}
この解法をホイン法(Heun method),あるいは修正Euler法(modified Euler method),2次のルンゲ・クッタ法と呼ばれることがある.

\subsection{ホイン法の誤差解析}
ホイン法の局所誤差および大域誤差を計算する.そのための準備として,再びTaylor展開を考え$\Delta t$の2次のオーダーまでとる.
\begin{equation}
	x(t+\Delta t)=x(t) +\frac{1}{1!}x^{\prime}\Delta t +\frac{1}{2!}x^{\prime\prime}\Delta t^2+\mathcal{O}((\Delta t)^3)
\end{equation}
さらに$dx(t)/dt=f(x,t)$と偏微分の公式より
\begin{align}\label{P}
	x^{\prime\prime}(t)&=\frac{d}{dt}f(x(t),t)
	=\frac{\partial }{\partial t}f(x(t),t)\frac{dt}{dt}+\frac{\partial }{\partial x}f(x(t),t)\frac{dx}{dt}\nn[10pt]
	&=\frac{\partial }{\partial t}f(x,t)+\frac{\partial }{\partial x}f(x,t)\frac{dx}{dt}
\end{align}
を得る.

ホインの差分公式\eqref{Heun}より,
\begin{align}
	x_{n+1}&=x_n+\frac{\Delta t}{2}\Biggl(
	f(x_n,t_n)
	+f(x_n+\Delta t f(x_n,t_n), t_n + \Delta t)
	\Biggr)
\end{align}
において,近似値$x_n$に真の解$x(t_n)$を代入すると,
\begin{align}
	x_{n+1}&=x(t_n)+\frac{\Delta t}{2}\Biggl(
	f(x(t_n),t_n)
	+f\biggl(
	x(t_n)+\Delta t f(x(t_n),t_n), t_n + \Delta t
	\biggr)
	\Biggr)
\end{align}
を得る.したがって局所誤差は
\begin{align}
	\epsilon_{n+1}&=x(t_{n+1})-x_{n+1}\nn[10pt]
	&=\frac{1}{1!}x^{\prime}(t_n)\Delta t +\frac{1}{2!}x^{\prime\prime}(t_n)\Delta t^2+\mathcal{O}((\Delta t)^3)\nn[10pt]
	&\hspace{20pt}-\frac{\Delta t}{2}\Biggl(
	f(x(t_n),t_n)
	+f\biggl(
	x(t_n)+\Delta t f(x(t_n),t_n), t_n + \Delta t
	\biggr)
	\Biggr)\nn[10pt]
	&=\frac{1}{1!}x^{\prime}(t_n)\Delta t +\frac{1}{2!}x^{\prime\prime}(t_n)\Delta t^2+\mathcal{O}((\Delta t)^3)\nn[10pt]
	&\hspace{20pt}-\frac{\Delta t}{2}\Biggl(
	f(x(t_n),t_n)+
	f(x(t_i),t_i)+\Delta tx^{\prime\prime}(t_i) + \mathcal{O}((\Delta t)^2)
	\Biggr)\nn[10pt]
	%
	&=\frac{1}{1!}x^{\prime}(t_n)\Delta t +\frac{1}{2!}x^{\prime\prime}(t_n)\Delta t^2+\mathcal{O}((\Delta t)^3)\nn[10pt]
	&\hspace{20pt}-\frac{\Delta t}{2}\Biggl(
	2f(x(t_n),t_n)+\Delta tx^{\prime\prime}(t_i) + \mathcal{O}((\Delta t)^2)
	\Biggr)\nn[10pt]
	%
	&=\frac{1}{1!}x^{\prime}(t_n)\Delta t +\frac{1}{2!}x^{\prime\prime}(t_n)\Delta t^2+\mathcal{O}((\Delta t)^3)\nn[10pt]
	&\hspace{20pt}-\frac{\Delta t}{2}\Biggl(
	2x^{\prime}(t_n)+\Delta tx^{\prime\prime}(t_i) + \mathcal{O}((\Delta t)^2)
	\Biggr)\nn[10pt]
	%
	&=\frac{1}{1!}x^{\prime}(t_n)\Delta t +\frac{1}{2!}x^{\prime\prime}(t_n)\Delta t^2+\mathcal{O}((\Delta t)^3)\nn[10pt]
	&\hspace{10pt}-\Delta t x^{\prime}(t_n)
	-\frac{\Delta t^2}{2}x^{\prime\prime}(t_i) + \mathcal{O}((\Delta t)^3)
	= \mathcal{O}((\Delta t)^3)
\end{align}
となり,$\Delta t^3$に比例することがわかる.次に,$n$ステップ後に蓄積した誤差$\epsilon^{\rm{gb}}$を計算する.$N=n\Delta t$より,$n \propto \Delta t^{-1}$,$n=\mathcal{O}(\Delta t)$であるから,$\epsilon^{\rm{gb}}$は
\begin{equation}
	\epsilon^{\rm{gb}}=n\epsilon
	=\mathcal{O}(\Delta t^{-1})\mathcal{O}(\Delta t^3)
	=\mathcal{O}(\Delta t^2)
\end{equation}
と求まる.


また,2変数のテイラー展開より,
\begin{align}\label{taylor}
f(x+\alpha\Delta x,t+\beta\Delta t)
%
\simeq f(x,t)+\frac{\partial f(x,t)}{\partial x}\Delta x&+\dfrac{\partial f(x,t)}{\partial t}\Delta t
\end{align}
である.$\alpha=f(x(t_i),t_i)$,$\beta=1$とし,式\eqref{taylor}を\eqref{P}に代入して,
\begin{align}
	f(x(t_i)+f(x(t_i),t_i)\Delta t, t_i+\Delta t)
	&=f(x(t_i),t_i)+f(x(t_i),t_i)\frac{\partial }{\partial x}f(x(t_i),t_i)\Delta x\nn[10pt]
	&+\dfrac{\partial}{\partial y} f(x(t_i),t_i)\Delta t
	+\mathcal{O}((\Delta t)^2)\nn[10pt]
	&=f(x(t_i),t_i)+\Delta tx^{\prime\prime}(t_i) + \mathcal(O)((\Delta t)^2)
\end{align}
を得る.
最後の変形で,$dx(t)/dt=f(x,t)$より,$f(x(t_i),t_i)=x^{\prime}(t_i)$を使い,第2項と第3項を次のようにまとめた:
\begin{align}
	&f(x_(t_i),t_i)\frac{\partial }{\partial x}f(x(t_i),t_i)\Delta t
	+\dfrac{\partial}{\partial y} f(x_i,t_i)\Delta t\nn[10pt]
	&=\left(f(x(t_i),t_i)\frac{\partial }{\partial x}f(x(t_i),t_i)
	+\dfrac{\partial}{\partial t} f(x(t_i),t_i)
	\right)\Delta t\nn[10pt]
	&=
	\left(x^{\prime}(t_i)\frac{\partial }{\partial x}f(x(t_i),t_i)
	+\dfrac{\partial}{\partial t} f(x(t_i),t_i)
	\right)\Delta t\nn[10pt]
	&=x^{\prime\prime}(t_i)\Delta t
\end{align}

\section{4次のルンゲ・クッタ法}
\begin{align}
	x_{n+1}&=x_n+\frac{1}{6}(k_1+2k_2+2k_3+k_4)\\[10pt]
	k_1&=\Delta t f(x_n,t_n)\\[10pt]
	k2 &=\Delta t f\left(x_n+\frac{k_1}{2},t_n+\frac{\Delta t}{2}\right)\\[10pt]
	k_3 &=\Delta t f\left(x_n+\frac{k_2}{2},t_n+\frac{\Delta t}{2}\right)\\[10pt]
	k_4&=\Delta tf(x_n+k_3,t_n+\Delta t)
\end{align}
局所誤差は$\mathcal O(\Delta t^5)$,累積誤差は$\mathcal O(\Delta t^4)$となることがホイン法と同様に示せる.
%%
\section{偏微分の復習}
二変数関数$f(x,y)$において,$y$を固定し,$x$で微分することを偏微分といい,
\begin{align}
\label{e1}
\frac{\partial f(x,y)}{\partial x}:=\lim_{\Delta x\to 0}\frac{f(x+\Delta x,y)-f(x,y)}{\Delta x}
\end{align}
で定義する.すると,関数$f(x,y)$の微小変化は
\begin{align}
\label{e2}
f(x+\Delta x,y)-f(x,y)=\frac{\partial f(x,y)}{\partial x}\Delta x+O((\Delta x)^2)
\end{align}
とかける.\\
 次に,変数$(x,y)$が$(x+\Delta x,y+\Delta y)$となったときの二変数関数$f(x,y)$の変化を考える.そのために,まず$f(x+\Delta x,y)$を$y$のみの関数とみて,その微小変化を考える.すると,
\begin{align}
f(x+\Delta x,y+\Delta y)-f(x+\Delta x,y)=\frac{\partial f(x+\Delta x,y)}{\partial y}\Delta y+O((\Delta y)^2)\\[5pt]
\label{e3}
\therefore
f(x+\Delta x,y+\Delta y)=f(x+\Delta x,y)+\frac{\partial f(x+\Delta x,y)}{\partial y}\Delta y+O((\Delta y)^2)
\end{align}
となる.次に(\ref{e3})右辺第1項$f(x+\Delta x,y)$を$x$のみの関数とみて,(\ref{e2})を適用すると,
\begin{align}
f(x+\Delta x,y+\Delta y)
=&f(x,y)+\frac{\partial f(x,y)}{\partial x}\Delta x+O((\Delta x)^2)\notag\\[5pt]
&+\frac{\partial f(x+\Delta x,y)}{\partial y}\Delta y+O((\Delta y)^2)
\end{align}
となる.整理すると,
\begin{align}\label{e4}
f(x+\Delta x,y+\Delta y)
=f(x,y)+\frac{\partial f(x,y)}{\partial x}\Delta x&+\frac{\partial f(x+\Delta x,y)}{\partial y}\Delta y\notag\\[5pt]
&+O((\Delta y)^2)+O((\Delta x)^2)
\end{align}
となる.(\ref{e4})右辺の第3項において,$g(x,y):=\dfrac{\partial f(x,y)}{\partial y}$,$g(x+\Delta x,y):=\dfrac{\partial f(x+\Delta x,y)}{\partial y}$とおく.$g(x,y)$を$x$の関数とみて,(\ref{e2})を適用すると,
\begin{align}
\label{e5}
g(x+\Delta x,y)=g(x,y)+\frac{\partial g(x,y)}{\partial x}\Delta x+O((\Delta x)^2)
\end{align}
となる.$g(x,y)$をもとに戻すと,
\begin{align}
\label{e6}
\dfrac{\partial f(x+\Delta x,y)}{\partial y}=\dfrac{\partial f(x,y)}{\partial y}+\frac{\partial }{\partial x}
\left\{
\dfrac{\partial f(x,y)}{\partial y}
\right\}
\Delta x+O((\Delta x)^2)
\end{align}
となる.これを(\ref{e4})右辺の第3項に代入すると,
\begin{align}\label{e7}
f(x+\Delta x,y+\Delta y)
=f(x,y)+\frac{\partial f(x,y)}{\partial x}\Delta x&+
\left(\dfrac{\partial f(x,y)}{\partial y}+\frac{\partial }{\partial x}
\left\{
\dfrac{\partial f(x,y)}{\partial y}
\right\}
\Delta x+O((\Delta x)^2)\right)
\Delta y\notag\\[5pt]
&+O((\Delta y)^2)+O((\Delta x)^2)\notag\\[10pt]
%
=f(x,y)+\frac{\partial f(x,y)}{\partial x}\Delta x&+\dfrac{\partial f(x,y)}{\partial y}\Delta y+\frac{\partial }{\partial x}
\left\{
\dfrac{\partial f(x,y)}{\partial y}
\right\}\Delta x\Delta y\notag\\[5pt]
&
+O((\Delta x)^2))\Delta y+O((\Delta y)^2)+O((\Delta x)^2)\notag\\[5pt]
\end{align}
ここで,2次以上の微小量を無視すれば,
%
\begin{align}\label{e8}
f(x+\Delta x,y+\Delta y)
%
\simeq f(x,y)+\frac{\partial f(x,y)}{\partial x}\Delta x&+\dfrac{\partial f(x,y)}{\partial y}\Delta y
\end{align}
を得る.


\section{プログラム}
\begin{mylisting}[language=C]
/* ヘッダファイルのインクルード */
#include <stdio.h>

int main(int argc, char **argv) {
    // コマンドラインから変数を読みとる
    int a, b;
    scanf("%d %d", &a, &b);

    // 割り算
    const double c = (double)a / (double)b;
    printf("%d / %d = %f\n", a, b, c);
}
\end{mylisting}





 \begin{thebibliography}{99}
\item
{砂川重信(1991)
『量子力学』(岩波書店)}
\item
{高野文彦(1975)
『多体問題 (新物理学シリーズ 18)』(培風館)}
\end{thebibliography}
\end{document}