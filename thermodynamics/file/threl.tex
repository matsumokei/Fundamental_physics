\section{熱力学の設定}


 位置$\bm{r}$と運動量$\bm{p}$からなる空間を考える.このような空間を位相空間という.この位相空間における分布関数をきめる方程式であるBoltzmann方程式を導出するのが本書の目的である.
\section{示量変数と示強変数}
一般の力$X_i$は,一般のポテンシャル$U(x_i)$とその変数である,系の一般の座標(変位)$x_i$を用いて
\begin{align}\label{force}
X_i=-\frac{\partial U(x_i)}{\partial x_i}
\end{align}
ここで,$X_i$と$x_i$の添え字$i$は系の$i$番目の自由度を表す.
\footnote{%
例えば,3次元直交座標$(x,y,z)$を考えたとき,1番目の自由度が$x$,2番目が$y$,3番目が$z$と対応させると,位置は$(x,y,z)=(x_1,x_2,x_3)$と対応する.}\\
%%\\
 今質点を任意の方向に$d\bm{x}=(x_1,x_2,\cdots,x_n)$だけ変位させたときの力$\bm{X}=(X_1,X_2,\cdots,X_n)$のなす力学的な仕事$W_{\rm{mecha}}$は,
\begin{align}\label{work}
W_{\rm{mecha}}=-dU=\bm{X}\cdot\bm{x}=\displaystyle\sum_{i=1}^{n}X_idx_i,\ \ \ \ \ \therefore dU=-\displaystyle\sum_{i=1}^{n}X_idx_i
\end{align}
である.\\
 今度は気体に対する熱力学の第1法則を考える.圧力を$P$,体積$V$として,外界が系に加えた仕事は$-PdV$と書ける.また,温度を$T$,エントロピーを$S$として,系が外界から得た熱量は$TdS$と書ける.すると,一般のポテンシャルである内部エネルギー$U$の変化$dU$は
\begin{align}\label{thm1}
dU=-PdV+TdS
\end{align}
と書ける.これと似た形式をもつ物理量として,エンタルピー$H$を使えば,その変化は
\begin{align}\label{thm1e}
dH=-VdP+SdT
\end{align}
と表される.(\ref{thm1})の第1項は圧力$P$が一般の力,体積$V$が変位を表している.しかし,(\ref{thm1e})の第1項で体積$V$が一般の力,圧力$P$が変位を表すとは言い難い.一般の力と変位の区別を明確にするために示強変数,示量変数という概念が必要になる.


%
\paragraph{示強変数}
力らしい特質をもった物理量を示強変数という.具体的には,
\begin{itemize}
  \item 系の自由度からどれか1つを選び,その自由度を変化させ,それに対するポテンシャルを変化させる強さを表す.
  \item 二つの系が接触して平衡状態にあるとき,両方の系で同じ値をとる,つりあうような量.
\end{itemize}
を示強変数という.

%
\paragraph{示量変数}
変位らしい特質をもった物理量を示量変数という.具体的には,
\begin{itemize}
  \item 二つの系が平衡状態にあるとき,つまり示強変数が等しいとき,二つの量の和が全体としての量となる.
\end{itemize}
を示量変数という.




















%
\section{偏微分の復習}
二変数関数$f(x,y)$において,$y$を固定し,$x$で微分することを偏微分といい,
\begin{align}
\label{e1}
\frac{\partial f(x,y)}{\partial x}:=\lim_{\Delta x\to 0}\frac{f(x+\Delta x,y)-f(x,y)}{\Delta x}
\end{align}
で定義する.すると,関数$f(x,y)$の微小変化は
\begin{align}
\label{e2}
f(x+\Delta x,y)-f(x,y)=\frac{\partial f(x,y)}{\partial x}\Delta x+O((\Delta x)^2)
\end{align}
とかける.\\
 次に,変数$(x,y)$が$(x+\Delta x,y+\Delta y)$となったときの二変数関数$f(x,y)$の変化を考える.そのために,まず$f(x+\Delta x,y)$を$y$のみの関数とみて,その微小変化を考える.すると,
\begin{align}
f(x+\Delta x,y+\Delta y)-f(x+\Delta x,y)=\frac{\partial f(x+\Delta x,y)}{\partial y}\Delta y+O((\Delta y)^2)\\[5pt]
\label{e3}
\therefore
f(x+\Delta x,y+\Delta y)=f(x+\Delta x,y)+\frac{\partial f(x+\Delta x,y)}{\partial y}\Delta y+O((\Delta y)^2)
\end{align}
となる.次に(\ref{e3})右辺第1項$f(x+\Delta x,y)$を$x$のみの関数とみて,(\ref{e2})を適用すると,
\begin{align}
f(x+\Delta x,y+\Delta y)
=&f(x,y)+\frac{\partial f(x,y)}{\partial x}\Delta x+O((\Delta x)^2)\notag\\[5pt]
&+\frac{\partial f(x+\Delta x,y)}{\partial y}\Delta y+O((\Delta y)^2)
\end{align}
となる.整理すると,
\begin{align}\label{e4}
f(x+\Delta x,y+\Delta y)
=f(x,y)+\frac{\partial f(x,y)}{\partial x}\Delta x&+\frac{\partial f(x+\Delta x,y)}{\partial y}\Delta y\notag\\[5pt]
&+O((\Delta y)^2)+O((\Delta x)^2)
\end{align}
となる.(\ref{e4})右辺の第3項において,$g(x,y):=\dfrac{\partial f(x,y)}{\partial y}$,$g(x+\Delta x,y):=\dfrac{\partial f(x+\Delta x,y)}{\partial y}$とおく.$g(x,y)$を$x$の関数とみて,(\ref{e2})を適用すると,
\begin{align}
\label{e5}
g(x+\Delta x,y)=g(x,y)+\frac{\partial g(x,y)}{\partial x}\Delta x+O((\Delta x)^2)
\end{align}
となる.$g(x,y)$をもとに戻すと,
\begin{align}
\label{e6}
\dfrac{\partial f(x+\Delta x,y)}{\partial y}=\dfrac{\partial f(x,y)}{\partial y}+\frac{\partial }{\partial x}
\left\{
\dfrac{\partial f(x,y)}{\partial y}
\right\}
\Delta x+O((\Delta x)^2)
\end{align}
となる.これを(\ref{e4})右辺の第3項に代入すると,
\begin{align}\label{e7}
f(x+\Delta x,y+\Delta y)
=f(x,y)+\frac{\partial f(x,y)}{\partial x}\Delta x&+
\left(\dfrac{\partial f(x,y)}{\partial y}+\frac{\partial }{\partial x}
\left\{
\dfrac{\partial f(x,y)}{\partial y}
\right\}
\Delta x+O((\Delta x)^2)\right)
\Delta y\notag\\[5pt]
&+O((\Delta y)^2)+O((\Delta x)^2)\notag\\[10pt]
%
=f(x,y)+\frac{\partial f(x,y)}{\partial x}\Delta x&+\dfrac{\partial f(x,y)}{\partial y}\Delta y+\frac{\partial }{\partial x}
\left\{
\dfrac{\partial f(x,y)}{\partial y}
\right\}\Delta x\Delta y\notag\\[5pt]
&
+O((\Delta x)^2))\Delta y+O((\Delta y)^2)+O((\Delta x)^2)\notag\\[5pt]
\end{align}
ここで,2次以上の微小量を無視すれば,
%
\begin{align}\label{e8}
f(x+\Delta x,y+\Delta y)
%
\simeq f(x,y)+\frac{\partial f(x,y)}{\partial x}\Delta x&+\dfrac{\partial f(x,y)}{\partial y}\Delta y
\end{align}
を得る.










%%
\section{Boltzmann方程式}
分布関数$f$を位置$\bm{r}(t)=(r_1(t),r_2(t),r_3(t))$,運動量$\bm{p}(t)=(p_1(t),p_2(t),p_3(t))$,時間$t$の関数$f(\bm{r}(t),\bm{p}(t),t)$の関数とする.時間$\Delta t$経過後の時刻$t+\Delta t$を$t^\prime=t+\Delta t$とおく.そして,時刻$t^\prime$における位置$\bm{r}(t^\prime)$を${\bm{r}}^\prime=\bm{r}(t^\prime)$とおく.また,時刻$t^\prime$における運動量$\bm{p}(t^\prime)$を${\bm{p}}^\prime=\bm{p}(t^\prime)$とおく.ここで,$\bm{r}^\prime,\bm{p}^\prime$がTaylar展開の1次までの項を用いて
\begin{align}\label{b1}
\bm{r}^\prime=\bm{r}(t^\prime)=\bm{r}(t+\Delta t)=\bm{r}(t)+\dfrac{d\bm{r}(t)}{dt}\Delta t=\bm{r}(t)+\Delta \bm{r}(t)\\[10pt]
\label{b2}
\bm{p}^\prime=\bm{p}(t^\prime)=\bm{p}(t+\Delta t)=\bm{p}(t)+\dfrac{d\bm{p}(t)}{dt}\Delta t=\bm{p}(t)+\Delta \bm{p}(t)
\end{align}
と書けることに注意したい.上の2式において,$\Delta \bm{r}(t):=\dfrac{d\bm{r}(t)}{dt}\Delta t$,$\Delta \bm{p}(t):=\dfrac{d\bm{p}(t)}{dt}\Delta t$とおいた.\\
 すると,変数$(\bm{r},\bm{p},t)$が$(\bm{r}^\prime,\bm{p}^\prime,t^\prime)$と変化したときの分布関数$f$の変化は
 \begin{align}\label{b3}
 &f(\bm{r}^\prime,\bm{p}^\prime,t^\prime)-f(\bm{r},\bm{p},t)=f(\bm{r}+\Delta \bm{r},\bm{p}+\Delta \bm{p},t+\Delta t)-f(\bm{r},\bm{p},t)\notag
 \intertext{(\ref{e8})より,}\notag\\
 &=\frac{\partial f(\bm{r},\bm{p},t)}{\partial r_1}\Delta r_1
+\frac{\partial f(\bm{r},\bm{p},t)}{\partial r_2}\Delta r_2
+\frac{\partial f(\bm{r},\bm{p},t)}{\partial r_3}\Delta r_3\notag\\[5pt]
&
\ \ \ +\frac{\partial f(\bm{r},\bm{p},t)}{\partial p_1}\Delta p_1
+\frac{\partial f(\bm{r},\bm{p},t)}{\partial p_2}\Delta p_2
+\frac{\partial f(\bm{r},\bm{p},t)}{\partial p_3}\Delta p_3
+\frac{\partial f(\bm{r},\bm{p},t)}{\partial t}\Delta t\notag
%
 \intertext{第1,2,3項と第4,5,6項を内積の形に直して}\notag\\
  &=
\left(
\frac{\partial f(\bm{r},\bm{p},t)}{\partial r_1}
,\frac{\partial f(\bm{r},\bm{p},t)}{\partial r_2},\frac{\partial f(\bm{r},\bm{p},t)}{\partial r_3}
\right)\cdot
(\Delta r_1,\Delta r_2,\Delta r_3)\notag\\[5pt]
&
\ \ \ +
\left(
\frac{\partial f(\bm{r},\bm{p},t)}{\partial p_1}
,\frac{\partial f(\bm{r},\bm{p},t)}{\partial p_2}
,\frac{\partial f(\bm{r},\bm{p},t)}{\partial p_3}
\right)
\cdot\left(\Delta p_1,\Delta p_2,\Delta p_3\right)
+\frac{\partial f(\bm{r},\bm{p},t)}{\partial t}\Delta t
\end{align}
と書ける.ここで,$\bm{r}$と$\bm{p}$についての微分演算子として,
 \begin{align}\label{nr}
 \nabla_{\bm{r}}&:=
\left(
\frac{\partial }{\partial r_1}
,\frac{\partial }{\partial r_2}
,\frac{\partial }{\partial r_3}
\right)\\[5pt]
\label{np}
 \nabla_{\bm{p}}&:=
\left(
\frac{\partial }{\partial p_1}
,\frac{\partial }{\partial p_2}
,\frac{\partial }{\partial p_3}
\right)
\end{align}
を導入する.また,
  \begin{align}\label{dr}
(\Delta r_1,\Delta r_2,\Delta r_3)
&=\left(
\dfrac{dr_1(t)}{dt}\Delta t,\dfrac{dr_2(t)}{dt}\Delta t,\dfrac{dr_3(t)}{dt}\Delta t
\right)\notag\\[5pt]
&=\frac{d\bm{r}(t)}{dt}\Delta t
\\[10pt]
%
\label{dp}
\left(\Delta p_1,\Delta p_2,\Delta p_3\right)
&=\left(
\dfrac{dp_1(t)}{dt}\Delta t,\dfrac{dp_2(t)}{dt}\Delta t,\dfrac{dp_3(t)}{dt}\Delta t
\right)\notag\\[5pt]
&=\frac{d\bm{p}(t)}{dt}\Delta t
\end{align}
であることを用いると,(\ref{b3})は
  \begin{align}\label{b4}
&f(\bm{r}^\prime,\bm{p}^\prime,t^\prime)-f(\bm{r},\bm{p},t)\notag\\[5pt]
&=
\biggl(
\nabla_{\bm{r}}f(\bm{r},\bm{p},t)
\biggr)
\cdot
\frac{d\bm{r}(t)}{dt}\Delta t
%
+
\biggl(
\nabla_{\bm{p}}f(\bm{r},\bm{p},t)
\biggr)
\cdot
\frac{d\bm{p}(t)}{dt}\Delta t
+
\frac{\partial f(\bm{r},\bm{p},t)}{\partial t}\Delta t
\end{align}
(\ref{b4})の両辺を$\Delta t$でわり,整理すれば,
%
\footnotesize
\begin{align}\label{b5}
\frac{\partial f(\bm{r},\bm{p},t)}{\partial t}
+
\frac{d\bm{r}(t)}{dt}
\cdot
\biggl(
\nabla_{\bm{r}}f(\bm{r},\bm{p},t)
\biggr)
%
+
\frac{d\bm{p}(t)}{dt}
\cdot
\biggl(
\nabla_{\bm{p}}f(\bm{r},\bm{p},t)
\biggr)
%
=
\frac
{f(\bm{r}^\prime,\bm{p}^\prime,t^\prime)-f(\bm{r},\bm{p},t)}{\Delta t}
\end{align}
\normalsize
さらに,速度$\bm{v}(t)=\dfrac{d\bm{r}(t)}{dt}$とニュートンの運動方程式$\dfrac{d\bm{p}(t)}{dt}=\dfrac{\bm{F}}{m}$を(\ref{b5})へ代入すると
\small
\begin{align}\label{bltzeq}
\frac{\partial f(\bm{r},\bm{p},t)}{\partial t}
+
\bm{v}(t)
\cdot
\biggl(
\nabla_{\bm{r}}f(\bm{r},\bm{p},t)
\biggr)
%
+
\dfrac{\bm{F}}{m}
\cdot
\biggl(
\nabla_{\bm{p}}f(\bm{r},\bm{p},t)
\biggr)
%
=
\frac
{f(\bm{r}^\prime,\bm{p}^\prime,t^\prime)-f(\bm{r},\bm{p},t)}{\Delta t}
\end{align}
\normalsize
の式を得る.この式は分布関数$f(\bm{r},\bm{p},t)$をきめるべき基本的な方程式であり,Boltzmann方程式という.(\ref{bltzeq})の右辺は分布関数の変化が生じる原因を示す項である.




 